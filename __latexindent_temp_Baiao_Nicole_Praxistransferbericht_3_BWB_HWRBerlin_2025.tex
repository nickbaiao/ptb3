% NICOLE CAMILLE BAIÃO - PRAXISTRANSFERBERICHT III

% === DOKUMENTKLASSE ===
\documentclass[
    12pt, 
    a4paper,
    ]{scrartcl} % KOMA-Script-Artikelklasse mit deutscher Typografie

% === SPRACHE & ZEICHENSATZ ===
\usepackage[ngerman]{babel} % Deutsche Sprache und Silbentrennung
\usepackage{csquotes} % Typografisch korrekte Anführungszeichen

% === SEITENLAYOUT & FORMATIERUNG ===
\usepackage[a4paper, margin=2.5cm]{geometry}
\usepackage{setspace}
\setstretch{1.15}
\usepackage{fancyhdr}
\usepackage{float} % Exakte Platzierung von Gleitobjekten (z. B. Bilder)
\usepackage{caption} % Bild-/Tabellenbeschriftungen anpassen
\usepackage{adjustbox} % Elemente (z. B. Tabellen) skalieren oder ausrichten
\usepackage{makecell} % Mehrzeilige Zellen in Tabellen (Tabellen in Tabellen)
\usepackage{ragged2e}
\justifying 
% === SCHRIFTARTEN ===
\usepackage{iftex} % Prüft, ob XeTeX oder LuaTeX verwendet wird
\usepackage{fontspec} % Benutzerdefinierte Schriftarten (nur mit XeLaTeX/LuaLaTeX)
\usepackage{titlesec}
% Aktuelle Größen beibehalten:
\titleformat{\section}{\normalfont\fontsize{14}{5}\bfseries}{\thesection}{1em}{}
\titleformat{\subsection}{\normalfont\fontsize{13}{5}\bfseries}{\thesubsection}{1em}{}
\titleformat{\subsubsection}{\normalfont\fontsize{12}{5}\bfseries}{\thesubsubsection}{1em}{} 

% ABSTÄNDE ANPASSEN — Das ist der Schlüssel:
\titlespacing*{\section}{0pt}{1.1\baselineskip}{0.1\baselineskip}
\titlespacing*{\subsection}{0pt}{1.0\baselineskip}{0.1\baselineskip}
\titlespacing*{\subsubsection}{0pt}{0.9\baselineskip}{0.1\baselineskip}
 % Hauptschriftart
\setmainfont{Times New Roman} % Setzt Times New Roman als Hauptschrift

% === GRAFIKEN & ABBILDUNGEN ===
\usepackage{graphicx} % Bilder einfügen
\usepackage{xcolor} % Farben verwenden (z. B. in Code oder Text)
\usepackage{longtable}
\usepackage{tablefootnote}
% === QUELLEN & VERWEISE ===
\usepackage{cite} % Quellenangaben
\usepackage[
    colorlinks,       % Links ohne Umrandungen in gewählter Farbe
    linkcolor=black,  % Farbe interner Verweise
    filecolor=black,  % Farbe externer Verweise
    citecolor=black,  % Farbe von Zitaten
    urlcolor=black,    % Farbe von Links
    breaklinks
]{hyperref} % Verlinkungen
\usepackage{cleveref} % Querverweise der Querverweise
\newcommand{\etalchar}[1]{#1}

% === TABELLEN ===
\usepackage{longtable} % Tabellen über mehrere Seiten
% makecell wurde oben bereits geladen

% === CODE-DARSTELLUNG ===
\usepackage{listings} % Quellcode einfügen und formatieren

%%%%%%%%%%%%%%%%%%%%%%%%%%

% === ÜBERSCHRIFTEN-FORMATIERUNG ===
\setkomafont{sectioning}{\rmfamily\bfseries} % Überschriften fett und serifenlos

% === VERLINKUNGEN & REFERENZEN ===
\hypersetup{breaklinks=true} % Erlaubt Zeilenumbruch bei langen Links (wichtig für URLs oder Literaturverweise)
\newcommand{\anhang}[1]{\hyperref[#1]{\nameref{#1}}} % Eigener Befehl für Anhang-Verweise mit automatisch lesbarem Namen (anstatt z. B. nur "Abschnitt 5.1")

% === SEITENLAYOUT & RÄNDER ===
\geometry{
    verbose,
    tmargin=3cm,  % Oberer Rand
    bmargin=2cm,  % Unterer Rand
    lmargin=2.1cm,  % Linker Rand
    rmargin=3cm   % Rechter Rand
}

% === TYPOGRAFISCHE FEINHEITEN ===
\displaywidowpenalty = 5000
\widowpenalty=5000
\clubpenalty=5000
% === FUßNOTEN UND VERWEISE ===
\newcommand{\footfigref}[1]{\footnote{Abb. \ref{#1} auf Seite \pageref{#1}}} % Eigener Fußnotenbefehl für Bildverweise mit Seitenzahl

% === ABSTÄNDE & ABSATZFORMATIERUNG ===
\setlength{\footskip}{10mm} % Abstand zwischen Text und Fußzeile
\setlength{\abovecaptionskip}{4pt}  % Abstand oberhalb von Bild-/Tabellenunterschriften
\setlength{\belowcaptionskip}{0pt}  % Abstand unterhalb von Bild-/Tabellenunterschriften
\setlength{\intextsep}{1pc}         % Abstand bei Gleitobjekten (z. B. Bilder im Textfluss)
\setlength{\parindent}{0pt} % Kein Einzug am Absatzanfang
\setlength{\parskip}{0.75\baselineskip} % Einheitlicher Abstand zwischen Absätzen
\setstretch{1.5} % 1½-zeiliger Zeilenabstand
% === BILD- UND TABELLENUNTERSCHRIFTEN VERKLEINERN ===
\addtokomafont{caption}{\small} % Macht Bild-/Tabellenunterschriften kleiner

% === INHALTSVERZEICHNIS & LISTEN ===
\KOMAoptions{listof=entryprefix, toc=indent} % Inhaltsverzeichnis: eingerückt und mit Präfix bei Abbildungs-/Tabellenlisten

% === CODEDARSTELLUNG (SQL) ===
\lstdefinelanguage{SQL}{
    keywords={SELECT, FROM, WHERE, INSERT, INTO, VALUES, UPDATE, SET, DELETE, JOIN, LEFT, RIGHT, INNER, ORDER, BY, GROUP, ASC, DESC, CREATE, TABLE, PRIMARY, KEY, FOREIGN, ALTER, DROP, DATABASE, BEGIN, TRANSACTION, IF, EXISTS, COMMIT, AND},
    keywordstyle=\color{blue},           % Keywords blau
    comment=[l]{--},                     % Kommentare mit --
    morecomment=[s]{/*}{*/},             % Blockkommentare
    commentstyle=\color{gray},          % Kommentare grau
    stringstyle=\color{red},            % Strings rot
}

\lstset{
    basicstyle=\ttfamily\linespread{0.85}\selectfont,   % Standard-Monospace-Schrift für Listings
    keywordstyle=\bfseries\color{blue},  % Beispiel-Stil für Keywords
    commentstyle=\itshape\color{gray},   % Stil für Kommentare
    stringstyle=\color{red},            % Stil für Strings
    columns=flexible,       % Flexibler Zeilenumbruch
    language=SQL,
    breaklines=true
}

%%%%%%%%%%%%%%%%%%%%%%%%%%

% === BEGINN DOKUMENT ===
\begin{document}
% === LITERATUR EINBINDEN (alle Einträge aus der .bib ins Literaturverzeichnis) ===
\nocite{*}

% --- Seitenzahlen-Formatierung 1/3: Römische Zahlen ---
\pagenumbering{Roman}  % Nutzt römische Zahlen (I, II, III, …)
\pagestyle{fancy}      % Aktiviert fancyhdr für benutzerdefinierte Kopf-/Fußzeilen
\fancyhf{}             % Setzt alle Kopf- und Fußzeilen zurück
\fancyhead[R]{\thepage}  % Seitenzahl oben rechts anzeigen
\renewcommand{\headrulewidth}{0pt} % Keine Linie in der Kopfzeile
\renewcommand{\footrulewidth}{0pt} % Keine Linie in der Fußzeile

%%%%%%%%%%%%%%%%%%%%%%%%%%

% === BEGINN TITELSEITE ===
\begin{titlepage}
\begingroup
\setstretch{1.15} % Nur auf der Titelseite engerer Zeilenabstand
\centering % Zentriert den gesamten Inhalt der Titelseite
   
% --- Logos einfügen (links und rechts) ---
\includegraphics[scale=0.15]{BILDER-PTB3/Berliner-wasserbetriebe.svg.png}\hfill % \hfill sorgt dafür, dass die beiden Logos links und rechts stehen
\includegraphics[scale=0.25]{BILDER-PTB3/Hochschule_für_Wirtschaft_und_Recht_Berlin_logo.jpg} \\[0.5cm] % \\[1.8cm] fügt einen vertikalen Abstand von 1.8 cm nach den Bildern ein

% --- Titel (fette und große Schrift) ---
\Large{\textbf{Handlungsempfehlungen zur Optimierung von Schulungs- und Kommunikationsprozessen für zukünftige Microsoft 365-Anwendungen - basierend auf Planner und To Do bei den Berliner Wasserbetriebe}} \\[0.4cm]

% --- Untertitel ---
{\Large Praxistransferbericht 3} \\

% --- Neuer Text mit Abstand --
\vspace{0.5cm} % Abstand zum Unterschift
\large % Setzt die Schriftgröße auf Standard
vorgelegt am 02.03.2026 an der\\ [0.2cm]
Hochschule für Wirtschaft und Recht Berlin\\ [0.2cm]
Fachbereich Duales Studium\\ [0.2cm]
\vspace{0.4cm} % Abstand zur Tabelle

% --- Tabelle mit persönlichen Angaben ---
\begin{tabular}{l l}  
    \textbf{\normalsize{von}} & \normalsize{Nicole Camille Baião}  \\[0.1cm]
    \textbf{\normalsize{Bereich:}} & \normalsize{Wirtschaft} \\[0.1cm]
    \textbf{\normalsize{Fachrichtung:}} & \normalsize{Wirtschaftsinformatik} \\[0.1cm]
    \textbf{\normalsize{Studienjahrgang:}} & \normalsize{2024} \\[0.1cm]
    \textbf{\normalsize{Studienhalbjahr:}} & \normalsize{3. Semester} \\[0.1cm]
    \textbf{\normalsize{Ausbildungsbetrieb:}} & \normalsize{Berliner Wasserbetriebe}  \\[0.1cm]
    \textbf{\normalsize{Betreuender Prüfer:}} & \normalsize{Prof. Dr. Gert Faustmann} \\[0.1cm]
    \textbf{\normalsize{Betreuer im Betrieb:}} & \normalsize{Diana Ireri Jimenez Ireta} \\[0.6cm]
\end{tabular} 

% --- Bereich für Unterschriften ---
\raggedright % Setzt den folgenden Text linksbündig
\textbf{\normalsize{Von der Ausbildungsleiterin zur Kenntnis genommen:}} \\[0.5cm]

% --- Tabelle mit Unterschriften ---
\begin{tabular}{p{8cm} p{8cm}}
    \normalsize{Berlin, den 27. Februar 2026} & \normalsize{Berlin, den 27. Februar 2026} \\[0.03cm]

    \includegraphics[width=5.7cm]{BILDER-PTB3/unterschrift_nicole.png} &
    \includegraphics[width=3.3cm]{BILDER-PTB3/unterschrift_diana.png} \\[0.05cm]

    \normalsize{\underline{\hspace{5cm}}} & \normalsize{\underline{\hspace{5cm}}} \\[0.05cm]
    \normalsize{(Unterschrift der Studentin)} & \normalsize{(Unterschrift des Ausbildungsleiters)}
\end{tabular}
% === ENDE TITELSEITE ===
\endgroup
\end{titlepage}

% === Inhaltsverzeichnis === 
\setcounter{page}{1} % Setzt die Seitennummer auf 1 für neue Zählung (Neustart der Zählung)
\newpage
\section*{Inhaltsverzeichnis}  
\renewcommand{\contentsname}{}
\vspace{-1.3cm}  % Anpassung des vertikalen Abstands
\tableofcontents % Inhaltsverzeichnis erstellen

% === Abküzungsverzeichnis === 
\newpage
\section*{Abkürzungsverzeichnis}  
\addcontentsline{toc}{section}{Abkürzungsverzeichnis}
\vspace{0.6cm}
\begin{tabular}{p{4cm} p{11.35cm}}   
    \textbf{\normalsize BWB:} & \normalsize Berliner Wasserbetriebe \\[0.5cm]
    \hline
    \textbf{\normalsize BBB:} & \normalsize BBBB \\[0.5cm]
    \hline
    \textbf{\normalsize CCC:} & \normalsize CCCC \\[0.5cm]
 \end{tabular}

% --- Seitenzahlen-Formatierung 2/3: Wechsel auf normale (arabische) Zahlen ---
\newpage
\pagenumbering{arabic} % Normale Zahlen ab "Einleitung"
\setcounter{page}{1}
% === Einleitung ===
\section{Einleitung}
Die Berliner Wasserbetriebe (BWB) sind das größte Wasser- und Abwasserversorgungsunternehmen Deutschlands und zählen zu den größten Arbeitsgeber in Berlin. Mit rund 4.800\footnote{Berliner Wasserbetriebe: \textit{Wir als Arbeitgeber}} Mitarbeitenden und zahlreichen Fachabteilungen agiert das Unternehmen in einem komplexen organisatorischen Umfeld mit hohen regulatorischen Anforderungen, in dem IT-Abteilungen eine zentrale Rolle bei der Unterstützung und Weiterentwicklung digitaler Arbeitsprozesse einnimmt.

Die Digitalisierung der Arbeitswelt hat in den letzen Jahren erheblich an Bedetung gewonnen, insbesondere durch die verstärke Einführung von Cloud-basierend Tools wie Microsoft 365-Anwendungen, darunter bei den Berliner Wasserbetrieben Microsoft Planner und Microsoft To Do. Diese Tools bieten vielfältige Möglichkeiten zur Verbesserung der Zusammenarbeit, Kommunikation und Produktivität innerhalb von Organisationen und sie stehen bei den BWB bereits zu Verfügung und werden in unterschliedlichen Abteilungen aktiv verwendet.

Die Planung, Steurung und Begleitung dieser Anwendungen erfolgt bei den BWB innerhalb der IT-Z/P (Methoden und Werkzeuge). Dort werden standardisierte Prozesse für Kommunkation und Schulung entwickelt, mit dem Ziel, eine einheitliche, nachvollziehbare nachhaltige Nutzung der bereitsgestellten Tools im Unternehmen sicherzustellen.

Ziel dieser Arbeit ist die Analyse der bestehenden Schulungs- und Kommunikationsprozessen im Zusammenhang mit der Nutzung von Microsoft Planner und To Do bei den BWB. Auf dieser Grundlage werden Handlungsempfehlungen für zufünftige Microsoft 365-Anwendungen, wie beispielweise Microsoft Forms, Lists oder neue Funktionen in Microsoft Teams, abgeleitet. Methodisch basiert die Arbeit auf Grundlagen des Geschäftsprozessmanagements (BPM). Bestehende IST-Prozesse werden analysiert und anhand geeigneter BPM-Kriterien bewertet, um daraus konzeptionelle SOLL-Prozesse abzueleiten, die als Orientierung für die zukünftige Gestaltung von Schulungs- und Kommunikationsprozessen dienen können. 
% === TEIL 1 ===
\section{Theoretischer Rahmen}
AAAA

\begin{itemize}
    \setlength\itemsep{0.4\baselineskip}
    \setlength\parskip{0pt}
    \setlength\parsep{0pt}
    
    \item \normalsize aaaa.
    \item \normalsize bbbb.
    \item \normalsize cccc.
  \end{itemize}

  % === TEIL 2 ===
\section{TEIL 2}
% --- TEIL 2.1 ---
\subsection{AAAAA}
AAAAAA
% --- TEIL 2.2 ---
\subsection{BBBB}

% === Glossar ====
\section*{Glossar}  
\addcontentsline{toc}{section}{Glossar} 
\vspace{0.6cm}
\begin{tabular}{p{4cm} p{11.35cm}}   
    \hline
    \textbf{\normalsize AAAA:} & \normalsize AAAAA.\footnotemark[70]\\[0.5cm]
    \hline
    \textbf{\normalsize BBBB:} & \normalsize BBBB. \footnotemark[71]\\[0.5cm]
\end{tabular}

% Fußnoten für Teil 1
% \footnotetext[70]{Vgl. \cite{quelle_namen}.}
% \footnotetext[71]{Vgl. \cite{name1212}, Abschnitt „XXXX“.}

% === Literaturverzeichnis ===
\newpage
\section*{Literaturverzeichnis} 
\addcontentsline{toc}{section}{Literaturverzeichnis}

% === Verweis auf Literaturverzeichnis ===
\renewcommand{\refname}{} % kein extra titel der quellen
\vspace{-\baselineskip} % Den zusätzlichen Abstand entfernen
\bibliographystyle{hwrbib}  % Wähle einen Stil (z.B. plain, alpha, unsrt, apalike, etc.)
\bibliography{quellen-ptb3}    % Verweis auf quellen.bib (ohne .bib-Endung)

% === Anhang ===
\newpage
\section*{Anhangsverzeichnis}
\addcontentsline{toc}{section}{Anhangsverzeichnis}
\vspace{0.6cm}
\begin{itemize}
    \item \hyperref[sec:Anhang1]{Anhang 1}, S. \pageref{sec:Anhang1}
    \item \hyperref[sec:Anhang2]{Anhang 2}, S. \pageref{sec:Anhang2}
\end{itemize}

\appendix
\renewcommand{\thesection}{\Alph{section}}
\newpage
% Anhang 1
\newpage
\phantomsection
\addcontentsline{toc}{subsection}{Anhang 1}
\label{sec:Anhang1}

\vspace*{\fill}
\begin{center}
    \begin{adjustbox}{rotate=90, center, valign=b}
        \begin{minipage}{0.9\textheight}
            \centering
            {\Large\bfseries Anhang 1}\\[1em]
            AAAA
        \end{minipage}
    \end{adjustbox}
\end{center}
\vspace*{\fill}

% === Ehrenwörtliche Erklärung ===
\newpage
\section*{Ehrenwörtliche Erklärung}
\addcontentsline{toc}{section}{Ehrenwörtliche Erklärung}

„Ich erkläre ehrenwörtlich:

\begin{itemize}
    \raggedright
    \item \normalsize dass ich meinen Praxistransferbericht ohne fremde Hilfe erstellt habe,
    \item \normalsize dass ich wörtliche Zitate aus der Literatur – etwa aus Büchern, wissenschaftlichen Artikeln oder offiziellen Quellen – entnommen und entsprechend gekennzeichnet habe,
    \item \normalsize dass ich Künstliche Intelligenz zur Formulierung einiger Sätze verwendet habe, und ebenfalls zur Korrektur der grammatikalischen Fehler beim Schreiben“
\end{itemize}
„Alle Quellen, die aus dem Internet oder aus anderen digitalen Medien stammen, die für den Praxistransferbericht verwendet wurden, sind der Arbeit beigefügt.“

„Ich bin mir bewusst, dass eine falsche Erklärung rechtliche Konsequenzen nach sich zieht“.\\

% --- Tabelle mit Unterschriften ---

\begin{tabular}{p{8cm}}
    \normalsize{Berlin, den XX. XXXX 202X} \\[0.5cm]
    \includegraphics[width=5.5cm]{BILDER-PTB3/unterschrift_nicole.png} \\
    \normalsize{\underline{\hspace{5.5cm}}} \\
    \normalsize{(Unterschrift der Studentin)}
\end{tabular} % \underline{\hspace{6cm}} erzeugt eine Linie mit einer Länge von 6 cm


% === ENDE DOKUMENT ===
\end{document}
